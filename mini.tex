\documentclass[ngerman,hyperref={pdfpagelabels=true}]{beamer}
\usepackage{etex}

% -----------------------------------------------------------------------------

\graphicspath{{img/}}

% -----------------------------------------------------------------------------

\usetheme{KIT}

\setbeamercovered{transparent}
\setbeamertemplate{enumerate items}[ball]

\usepackage[ngerman]{babel}
\usepackage[utf8]{inputenc}
\usepackage[TS1,T1]{fontenc}
\usepackage{array}
\usepackage[absolute,overlay]{textpos}
\usepackage{beamerKITdefs}
\usepackage{graphicx}
\usepackage{tikz}
\usepackage{listings}
\usepackage[loadonly]{enumitem}
\usepackage{hyperref}

\usetikzlibrary{trees,shapes}

\pdfpageattr {/Group << /S /Transparency /I true /CS /DeviceRGB>>}
\pdfpageattr {/Group << /S /Transparency /I true /CS /DeviceRGB>>}


\title{Pfadzerlegung zur Kompression von Tries}
\subtitle{
``Fast Compressed Tries through Path Decompositions'',
Grossi \& Ottaviano~\cite{tries} \\[1em]
Vorgestellt von Niklas Baumstark -- \textit{niklas.baumstark@gmail.com}
}


\author{Niklas Baumstark}
\institute{ITI-Sanders}

\TitleImage[width=\titleimagewd,height=\titleimageht]{titel.jpg}

\KITinstitute{ITI}
\KITfaculty{Fakultät für Informatik}

\lstset{
basicstyle=\ttfamily\normalsize
}

\newcommand{\tikzmark}[1]{\tikz[overlay,remember picture] \node (#1) {};}
\newcommand\T{\mathcal{T}}


% ----------------------------------------------------------------------
\begin{document}
% ----------------------------------------------------------------------

%\AtBeginSection[]
%{\frame{\frametitle{Outline}
%\tableofcontents[currentsection]}}

\setlength\textheight{7cm} %required for correct vertical alignment, if [t] is not used as documentclass parameter

% title frame
\begin{frame}
  \maketitle
\end{frame}


\begin{frame}{Pfadzerlegung}
\begin{itemize}
\item Anwendbar auf beliebigen Baum $\T$
\item Erzeuge neuen Baum $\T^c$
\item Knoten in $\T^c$ sind Pfade in $\T$
%\item Konstruktion:
%\begin{enumerate}
%\item W"ahle Pfad $p$ von Wurzel zu einem Blatt
%\item Entferne $p$
%\item Berechne rekursiv Dekompositionen $\T^c_1, \T^c_2, \ldots, \T^c_n$ der "ubrigen Teilb"aume,
%geordnet nach Level der Teilbaumwurzel in $T$
%\item $\T^c$ hat Wurzel $p$ mit Kindern $\T^c_1, \ldots, \T^c_n$.
%\end{enumerate}
\end{itemize}
\end{frame}

\begin{frame}{Pfadzerlegung (Beispiel)}

\begin{columns}[T] % align columns
\begin{column}{.48\textwidth}

\centering
$\T$\\[1em]
\begin{tikzpicture}[
  path11/.style={edge from parent/.style={black,thin,draw},circle,draw=black,thin,text=black},
  path00/.style={edge from parent/.style={black!20,thin,draw},circle,draw=black!20,thin,text=black!20},
  path00thick/.style={edge from parent/.style={black!20,very thick,draw},circle,draw=black!20,thin,text=black!20},
  path01/.style={edge from parent/.style={black!20,thin,draw},circle,draw=black,thin,text=black},
  path02/.style={edge from parent/.style={black!20,thin,draw},circle,draw=red,very thick,text=black},
  path22/.style={edge from parent/.style={red,very thick,draw},circle,draw=red,very thick,text=black},
  p1root/.style={path11}, p1/.style={path11},
  p2root/.style={path11}, p2/.style={path11},
  p3root/.style={path11}, p3/.style={path11},
  p4root/.style={path11}, p4/.style={path11},
  p5root/.style={path11}, p5/.style={path11},
  p6root/.style={path11}, p6/.style={path11},
]
\only<2->{\tikzset{
  p1root/.style={path22},
  p1/.style={path22},
}}
\only<3->{\tikzset{
  p1root/.style={path00},
  p1/.style={path00thick},
  p2root/.style={path01},
  p3root/.style={path01},
  p5root/.style={path01},
  p6root/.style={path01},
}}
\only<4->{\tikzset{
  p2root/.style={path02},
  p2/.style={path22},
}}
\only<5->{\tikzset{
  p2root/.style={path00},
  p2/.style={path00thick},
}}
\only<6->{\tikzset{
  p3root/.style={path02},
  p3/.style={path22},
}}
\only<7->{\tikzset{
  p3root/.style={path00},
  p3/.style={path00thick},
  p4root/.style={path01},
}}
\only<8->{\tikzset{
  p4root/.style={path02},
  p4/.style={path22},
}}
\only<9->{\tikzset{
  p4root/.style={path00},
  p4/.style={path00thick},
}}
\only<10->{\tikzset{
  p5root/.style={path02},
  p5/.style={path22},
}}
\only<11->{\tikzset{
  p5root/.style={path00},
  p5/.style={path00thick},
}}
\only<12->{\tikzset{
  p6root/.style={path02},
  p6/.style={path22},
}}
\only<13->{\tikzset{
  p6root/.style={path00},
  p6/.style={path00thick},
}}
\node[circle,draw,p1root](z){$a_1$}
  child[p1]{node[circle,draw]{$b_1$}
    child[p5root]{node[circle,draw]{$c_1$}}
    child[p1]{node[circle,draw]{$c_2$}
      child[p1]{node[circle,draw]{$d_1$}}
      child[p6root]{node[circle,draw]{$d_2$}}
    }
  }
  child[p2root]{node[circle,draw]{$b_2$}}
  child[p3root]{node[circle,draw]{$b_3$}
    child[p3]{node[circle,draw]{$c_3$}}
    child[p4root]{node[circle,draw]{$c_4$}}
  };
\end{tikzpicture}


\end{column}%
\hfill%
\begin{column}{.48\textwidth}

\centering
$\T^c$\\[1em]
\begin{tikzpicture}[
  inactive/.style={rectangle,draw,black,text=black,thin,
    edge from parent/.style={draw,black,thin}},
  active/.style={inactive,draw=red,very thick},
  invisible/.style={inactive,draw=white,text=white,edge from parent/.style={white}},
  p1/.style={invisible},
  p2/.style={invisible},
  p3/.style={invisible},
  p4/.style={invisible},
  p5/.style={invisible},
  p6/.style={invisible},
]
\only<2->{\tikzset{
  p1/.style={active},
}}
\only<3->{\tikzset{
  p1/.style={inactive},
}}
\only<4->{\tikzset{
  p2/.style={active},
}}
\only<5->{\tikzset{
  p2/.style={inactive},
}}
\only<6->{\tikzset{
  p3/.style={active},
}}
\only<7->{\tikzset{
  p3/.style={inactive},
}}
\only<8->{\tikzset{
  p4/.style={active},
}}
\only<9->{\tikzset{
  p4/.style={inactive},
}}
\only<10->{\tikzset{
  p5/.style={active},
}}
\only<11->{\tikzset{
  p5/.style={inactive},
}}
\only<12->{\tikzset{
  p6/.style={active},
}}
\only<13->{\tikzset{
  p6/.style={inactive},
}}
\node[draw,p1](z){$a_1b_1c_2d_1$}
  child[p2]{node[draw]{$b_2$}}
  child[p3]{
    node[draw]{$b_3c_3$}
    child[p4]{node[draw]{$c_4$}}
  }
  child[p5]{node[draw]{$c_1$}}
  child[p6]{node[draw]{$d_2$}};
\end{tikzpicture}


\begin{enumerate}
\onslide<2->{
  \item W"ahle Pfad $p$ von Wurzel zu einem Blatt
}
\onslide<3->{
  \item Entferne $p$
}
\onslide<4->{
  \item Berechne rekursiv Dekomposition der Teilb"aume
}
\end{enumerate}

\end{column}%
\end{columns}

\end{frame}

\begin{frame}{Pfadzerlegung}
\begin{itemize}
\item Wie entscheiden, welcher Pfad zu w"ahlen ist?
\item Zwei grundlegende Taktiken
\begin{itemize}
\item Lexikografisch: Immer linkestes Kind w"ahlen
\item Heavy-Light: Folge \emph{schweren Kanten} in gr"o"sten Teilbaum
\\$\rightarrow \T^c$ hat H"ohe $h = \mathcal{O}(\log |\T|)$
\end{itemize}
\end{itemize}
\end{frame}

\newcommand{\edgelabel}[2]{edge from parent node[draw=none,#1]{\lstinline|#2|}}

\begin{frame}{Pfadzerlegung eines Tries}
\begin{itemize}
\item Betrachte Patricia-Trie mit Knotenlabels
\end{itemize}

\begin{columns}[T] % align columns
\begin{column}{.48\textwidth}

\centering
$\T$ \\[1em]
\begin{tikzpicture}[
  every node/.style={rectangle,draw,black,text=black,thin,
    edge from parent/.style={draw,black,thin}},
  every child/.style={level distance=1.25cm},
]
\node(z){\lstinline|r|}
  child{
    node{\lstinline|om|}
    child{
      node{\lstinline|ulus|}
      \edgelabel{left}{u}
    }
    child{node{\lstinline|an|}
      child{
        node {\lstinline|e|}
        \edgelabel{left}{e}
      }
      child{
        node{\lstinline|us|}
        \edgelabel{right}{u}
      }
      \edgelabel{right}{a}
    }
    \edgelabel{left}{o}
  }
  child{
    node{\lstinline|emus|}
    \edgelabel{left}{e}
  }
  child{node{\lstinline|ub|}
    child{
      node{\lstinline|en|}
      \edgelabel{left}{e}
    }
    child{
      node{\lstinline|icon|}
      \edgelabel{right}{i}
    }
    \edgelabel{right}{u}
  };
\end{tikzpicture}


\end{column}%
\hfill%
\begin{column}{.48\textwidth}

%\centering
$\T^c$ \\
\begin{tikzpicture}[
  every node/.style={rectangle,draw,black,text=black,thin},
  every child/.style={edge from parent/.style={draw=none}, level distance = 1.25cm},
  noedge/.style={edge from parent/.style={white}},
  edge/.style={edge from parent/.style={black,thin,draw}},
  split/.style={rectangle split,rectangle split horizontal},
  conn/.style={out=270,in=90},
]
\node[split, rectangle split parts=4](a){
  \lstinline|r|
  \nodepart{two} \lstinline|om|
  \nodepart{three} \lstinline|an|
  \nodepart{four} \lstinline|e|
  }
  child{
    node(b){\lstinline|emus|}
  }
  child{
    node[split,rectangle split parts=2](c){
    \lstinline|ub|
    \nodepart{two} \lstinline|en|
    }
    child{
      node(f){\lstinline|icon|}
    }
  }
  child{
    node(d){\lstinline|ulus|}
  }
  child{
    node(e){\lstinline|us|}
  }
  ;

  \draw[conn] (a.one split south) to node [draw=none, midway, above] {\lstinline|e|} (b.north);
  \draw[conn] (a.one split south) to node [draw=none, midway, right] {\lstinline|u|} (c.north);
  \draw[conn] (a.two split south) to node [draw=none, midway, below] {\lstinline|u|} (d.north);
  \draw[conn] (a.three split south) to node [draw=none, midway, above] {\lstinline|u|} (e.north);
  \draw[conn] (c.one split south) to node [draw=none, midway, left] {\lstinline|i|} (f.north);
\end{tikzpicture}


\lstinline|remus| \\
\lstinline|romane| \\
\lstinline|romanus| \\
\lstinline|romulus| \\
\lstinline|ruben| \\
\lstinline|rubicon| \\

\end{column}
\end{columns}

\end{frame}

\begin{frame}{DFUDS}
\begin{itemize}
\item \emph{Depth-first unary degree sequence}
\item Kodiert Baum mit $2n$ Bits
\item Betrachte Knoten $x$ mit $c$ Kindern
\item Teilbaum $x$ wird kodiert als $c$ "offnende Klammern \lstinline|(| +
schlie"sende Klammer \lstinline|)| + Kodierungen der Kindb"aume
%\item Fast balanciert, bis auf \lstinline|)| der Wurzel. F"uge daher einzelnes
%\lstinline|(| am Anfang ein
\end{itemize}
\vspace{1em}
\begin{columns}

\begin{column}{0.34\textwidth}
\centering
\begin{tikzpicture}
\node[circle,draw](z){$1$}
  child{node[circle,draw]{$2$}
    child{node[circle,draw]{$3$}}
    child{node[circle,draw]{$4$}}
  }
  child{node[circle,draw]{$5$}}
  child{node[circle,draw]{$6$}}
  ;
\end{tikzpicture}
\end{column}

\begin{column}{0.6\textwidth}

\centering
\[ \begin{array}{lllllllll}
Knoten &=&&1&2&3&4&5 \\
BP &=
&\text{\lstinline|(|}
&\text{\lstinline|((()|}
&\text{\lstinline|(()|}
&\text{\lstinline|)|}
&\text{\lstinline|)|}
&\text{\lstinline|)|}
&\text{\lstinline|)|} \\
Index &=&0&1&5&8&9&10&11
\end{array}
\]

\end{column}

\end{columns}

\end{frame}

\begin{frame}{DFUDS -- Operationen}

\begin{columns}

\begin{column}{0.34\textwidth}
\centering
\begin{tikzpicture}
\node[circle,draw](z){$1$}
  child{node[circle,draw]{$2$}
    child{node[circle,draw]{$3$}}
    child{node[circle,draw]{$4$}}
  }
  child{node[circle,draw]{$5$}}
  child{node[circle,draw]{$6$}}
  ;
\end{tikzpicture}
\end{column}

\begin{column}{0.6\textwidth}

\centering
\[ \begin{array}{lllllllll}
Knoten &=&&1&2&3&4&5 \\
BP &=
&\text{\lstinline|(|}
&\text{\lstinline|((()|}
&\text{\lstinline|(()|}
&\text{\lstinline|)|}
&\text{\lstinline|)|}
&\text{\lstinline|)|}
&\text{\lstinline|)|} \\
Index &=&0&1&5&8&9&10&11
\end{array}
\]

\end{column}

\end{columns}

\vspace{1em}

\begin{itemize}
\item Finde $k$-tes Kind f"ur gegebenes $0 \le k < c$
\begin{itemize}
\item Finde $k - 1$-tes \lstinline|(| mit \lstinline|rank|/\lstinline|select|
\item Finde zugeh"origes \lstinline|)| mit \lstinline|FindClose|
\end{itemize}
\end{itemize}
\end{frame}

\begin{frame}{Trie-Repr"asentation}

\begin{itemize}

\item Repr"asentiere Baumstruktur von $\T^c$ als DFUDS
\item F"uge spezielle Splitter-Zeichen in Knotenlabels ein
\end{itemize}

\centering
\begin{tikzpicture}[
  every node/.style={rectangle,draw,black,text=black,thin},
  every child/.style={edge from parent/.style={draw=none}, level distance = 1.25cm},
  noedge/.style={edge from parent/.style={white}},
  edge/.style={edge from parent/.style={black,thin,draw}},
  split/.style={rectangle split,rectangle split horizontal},
  conn/.style={out=270,in=90},
  subtree/.style={isosceles triangle,shape border rotate=90,yshift=-0.6cm},
]
\node[split, rectangle split parts=4](a){
  \lstinline|r|
  \nodepart{two} \lstinline|om|
  \nodepart{three} \lstinline|an|
  \nodepart{four} \lstinline|e|
  }
  child{
    node[subtree](b){$\alpha$}
  }
  child{
    node[subtree](c){$\beta$}
  }
  child{
    node[subtree](d){$\gamma$}
  }
  child{
    node[subtree](e){$\delta$}
  }
  ;

  \draw[conn] (a.one split south) to node [draw=none, midway, above] {\lstinline|e|} (b.north);
  \draw[conn] (a.one split south) to node [draw=none, midway, right] {\lstinline|u|} (c.north);
  \draw[conn] (a.two split south) to node [draw=none, midway, below] {\lstinline|u|} (d.north);
  \draw[conn] (a.three split south) to node [draw=none, midway, above] {\lstinline|u|} (e.north);
\end{tikzpicture}

\[ \begin{array}{llllllllllll}
L &= &\text{\lstinline|r|} &\text{\textbf{2}}
     &&\text{\lstinline|om|} &\text{\textbf{1}}
     &\text{\lstinline|an|} &\text{\textbf{1}}
     &\text{\lstinline|e|}
     &&L_{\alpha\ldots \delta} \\
BP &= &&\text{\lstinline|(|}&\text{\lstinline|(|}
      &&\text{\lstinline|(|}
      &&\text{\lstinline|(|}
      &&\text{\lstinline|)|}
      &BP_{\alpha\ldots \delta} \\
B &= &&\text{\lstinline|e|}&\text{\lstinline|u|}
     &&\text{\lstinline|u|}
     &&\text{\lstinline|u|}
     &&&B_{\alpha\ldots \delta}
\end{array}
\]


\end{frame}

\begin{frame}
\centering
\begin{tikzpicture}[
  every node/.style={rectangle,draw,black,text=black,thin},
  every child/.style={edge from parent/.style={draw=none}, level distance = 1.25cm},
  noedge/.style={edge from parent/.style={white}},
  edge/.style={edge from parent/.style={black,thin,draw}},
  split/.style={rectangle split,rectangle split horizontal},
  conn/.style={out=270,in=90},
  subtree/.style={isosceles triangle,shape border rotate=90,yshift=-0.6cm},
]
\node[split, rectangle split parts=4](a){
  \lstinline|r|
  \nodepart{two} \lstinline|om|
  \nodepart{three} \lstinline|an|
  \nodepart{four} \lstinline|e|
  }
  child{
    node[subtree](b){$\alpha$}
  }
  child{
    node[subtree](c){$\beta$}
  }
  child{
    node[subtree](d){$\gamma$}
  }
  child{
    node[subtree](e){$\delta$}
  }
  ;

  \draw[conn] (a.one split south) to node [draw=none, midway, above] {\lstinline|e|} (b.north);
  \draw[conn] (a.one split south) to node [draw=none, midway, right] {\lstinline|u|} (c.north);
  \draw[conn] (a.two split south) to node [draw=none, midway, below] {\lstinline|u|} (d.north);
  \draw[conn] (a.three split south) to node [draw=none, midway, above] {\lstinline|u|} (e.north);
\end{tikzpicture}

\[ \begin{array}{llllllllllll}
L &= &\text{\lstinline|r|} &\text{\textbf{2}}
     &&\text{\lstinline|om|} &\text{\textbf{1}}
     &\text{\lstinline|an|} &\text{\textbf{1}}
     &\text{\lstinline|e|}
     &&L_{\alpha\ldots \delta} \\
BP &= &&\text{\lstinline|(|}&\text{\lstinline|(|}
      &&\text{\lstinline|(|}
      &&\text{\lstinline|(|}
      &&\text{\lstinline|)|}
      &BP_{\alpha\ldots \delta} \\
B &= &&\text{\lstinline|e|}&\text{\lstinline|u|}
     &&\text{\lstinline|u|}
     &&\text{\lstinline|u|}
     &&&B_{\alpha\ldots \delta}
\end{array}
\]


\begin{itemize}
\item Gegeben Zeichenkette $P$, finde Position von $P$ im zerlegten Trie $T^c$
\item Algorithmus:
\begin{itemize}
\item Scanne Label der Wurzel bis zum ersten Mismatch eines branching characters.
Sei $n$ die Summe vorhergehender Splitter und $m$ der aktuelle Splitter.
\item Bin"arsuche in $B[n \ldots n+m-1]$, um richtiges Kind zu finden
\item Rekursiv fortfahren
\end{itemize}

\end{itemize}
\end{frame}

\begin{frame}
\begin{block}{Lookup}
\begin{itemize}
\item Laufzeit $\mathcal{O}\left(|P| + h\log |\Sigma|\right)$
\item Nur $h$ beliebige Zugriffe auf $L$, ansonsten sequentiell
\item[$\implies$] Sehr cache-effizient, effektive Kompression m"oglich
\end{itemize}
\end{block}

\begin{block}{Weitere Anwendungen}
\begin{itemize}
\item Minimal Perfect Hashing: Labels (bis auf branching characters) weglassen, daf"ur
Tiefendeltas speichern. Strings werden abgebildet auf Knotennummer
\item Bei lexikografischer Zerlegung sogar monoton
%\item Tiefengarantie auch m"oglich
%\begin{itemize}
%\item Spezielle Variante von Heavy-Light-Zerlegung die lexikographische Ordnung erh"alt
%\item Laufzeit $\mathcal{O}\left(\min(|P|, \log n \cdot \log |\Sigma|)\right)$
%\end{itemize}
\end{itemize}
\end{block}
\end{frame}

\begin{frame}{Heavy-Light-Zerlegung mit lexikografischen Ordnung}
\begin{itemize}
\item Kodiere Alphabet bin"ar, $\log |\Sigma|$ Bits pro Zeichen
\item Konkateniere Kindb"aume in lexikografischer Ordnung
\end{itemize}

\begin{columns}

\tikzset{
every child/.style={edge from parent/.style={draw=none}},
subtree/.style={isosceles triangle,shape border rotate=90,yshift=-0.6cm},
}
\begin{column}{0.34\textwidth}
\centering
$\T$ \\[0.5em]
\begin{tikzpicture}
\node[circle,draw=red,thick](a1){$\delta_1$}
  child{node[circle,draw=red,thick](a2){$\delta_2$}
    child[subtree]{node[draw](b){b}}
    child{node[circle,draw=red,thick](a3){$\delta_3$}
      child{node[circle,thick,draw=red](a4){$\delta_4$}}
      child{node[circle,draw](c){c}}
    }
  }
  child[subtree]{node[draw](d){d}}
  ;

  \draw[draw=red,thick,color=red] (a1) to node [draw=none, midway, left, thick, yshift=0.2em] {\lstinline|0|} (a2);
  \draw (a2) to node [draw=none, midway, left, yshift=0.2em] {\lstinline|0|} (b.north);
  \draw[draw=red,thick,color=red] (a2) to node [draw=none, midway, right, yshift=0.2em] {\lstinline|1|} (a3);
  \draw (a3) to node [draw=none, midway, right, yshift=0.2em] {\lstinline|1|} (c);
  \draw (a1) to node [draw=none, midway, right, yshift=0.2em] {\lstinline|1|} (d.north);
  \draw[draw=red,thick,color=red] (a3) to node [draw=none, midway, left, yshift=0.2em] {\lstinline|0|} (a4);
\end{tikzpicture}
\end{column}

\begin{column}{0.6\textwidth}

\centering
$\T^c$ \\[1em]
\begin{tikzpicture}[
  every node/.style={rectangle,draw,black,text=black,thin},
  every child/.style={edge from parent/.style={draw=none}, level distance = 1.25cm},
  noedge/.style={edge from parent/.style={white}},
  edge/.style={edge from parent/.style={black,thin,draw}},
  conn/.style={},
  subtree/.style={isosceles triangle,shape border rotate=90,yshift=-0.6cm},
]
\node[circle,inner sep=0,minimum size=1.5em,draw=red,thick](a){}
  child{
    node[subtree](b){b}
  }
  child{
    node[circle](c){c}
  }
  child{
    node[subtree](d){d}
  }
  ;

  \draw[conn] (a) to node [draw=none, midway, above] {\lstinline|0|} (b.north);
  \draw[conn] (a) to node [draw=none, midway, right] {\lstinline|1|} (c);
  \draw[conn] (a) to node [draw=none, midway, above] {\lstinline|1|} (d.north);
\end{tikzpicture}

\centering

\[ \begin{array}{lllllllllllll}
L &=
&\delta_1 &\tikzmark{aa}\text{\textbf{0}}
&\delta_2 &\tikzmark{bb}\text{\textbf{1}}
&\delta_3 &\tikzmark{cc}\text{\textbf{1}}
\\
\\
BP &=
&&\tikzmark{dd}\text{\lstinline|(|}
&&\tikzmark{ee}\text{\lstinline|(|}
&&\tikzmark{ff}\text{\lstinline|(|}
&&\text{\lstinline|)|}
\\
Kind &=&&b&&c&&d
\end{array}
\]

\begin{tikzpicture}[overlay,remember picture,-latex,shorten >=5pt]
  \draw[distance=0.45cm,out=-10,in=130,yshift=1em,shorten <=5pt,shorten >=1pt] (aa.south) to (ff.north);
  \draw[distance=0.45cm,out=200,in=70] (bb.south) to (dd.north);
  \draw[distance=0.45cm,out=200,in=70] (cc.south) to (ee.north);
\end{tikzpicture}

\end{column}

\end{columns}

\end{frame}

\begin{frame}{Heavy-Light-Zerlegung mit lexikografischen Ordnung}
\begin{itemize}
\item Lookup wie zuvor, allerdings ist Knotennummer nicht lexikografischer Rang
\item Ziehe "ubersprungene linke Teilb"aume ab
\item Laufzeit $\mathcal{O}(\min\{|P|, \log n \cdot \log |\Sigma|\})$
\end{itemize}

\end{frame}

\begin{frame}{Referenzen}
\bibliographystyle{abbrv}
\bibliography{references}
\end{frame}

\end{document}
