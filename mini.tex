\documentclass[ngerman,hyperref={pdfpagelabels=true}]{beamer}
\usepackage{etex}

% -----------------------------------------------------------------------------

\graphicspath{{img/}}

% -----------------------------------------------------------------------------

\usetheme{KIT}

\setbeamercovered{transparent}
\setbeamertemplate{enumerate items}[ball]

\usepackage[ngerman]{babel}
\usepackage[utf8]{inputenc}
\usepackage[TS1,T1]{fontenc}
\usepackage{array}
\usepackage{multicol}
\usepackage[absolute,overlay]{textpos}
\usepackage{beamerKITdefs}
\usepackage{wrapfig}
\usepackage{graphicx}
\usepackage{tikz}
\usepackage{listings}
\usetikzlibrary{calc,trees,positioning,arrows,chains,shapes.geometric,%
    decorations.pathreplacing,decorations.pathmorphing,shapes,%
    matrix,shapes.symbols}

\pdfpageattr {/Group << /S /Transparency /I true /CS /DeviceRGB>>}


\pdfpageattr {/Group << /S /Transparency /I true /CS /DeviceRGB>>}


\title{Pfadzerlegung zur Kompression von Tries}
\subtitle{Niklas Baumstark -- \textit{niklas.baumstark@gmail.com}}


\author{Niklas Baumstark}
\institute{ITI-Sanders}

\TitleImage[width=\titleimagewd,height=\titleimageht]{titel.jpg}

\KITinstitute{ITI}
\KITfaculty{Fakultät für Informatik}

\tikzstyle{vertex}=[circle,minimum size=17pt,inner sep=0pt,draw=black]
\tikzstyle{small-vertex}=[circle,minimum size=5pt,inner sep=0pt,draw=black]
\tikzstyle{edge} = [stealth,draw,bend left,thick,in=20,out=20]
\tikzstyle{weight} = [font=\small]
\tikzstyle{array-element}=[rectangle, draw=black, minimum size=11.3333pt]

\lstset{
basicstyle=\ttfamily\small
}

% ----------------------------------------------------------------------
\begin{document}
% ----------------------------------------------------------------------

%\AtBeginSection[]
%{\frame{\frametitle{Outline}
%\tableofcontents[currentsection]}}

\setlength\textheight{7cm} %required for correct vertical alignment, if [t] is not used as documentclass parameter

% title frame
\begin{frame}
  \maketitle
\end{frame}

\tikzset{
>=stealth',
  punktchain/.style={
    rectangle,
    rounded corners,
    % fill=black!10,
    draw=black, very thick,
    text width=10em,
    minimum height=3em,
    text centered,
    on chain},
  line/.style={draw, thick, <-},
  element/.style={
    tape,
    top color=white,
    bottom color=blue!50!black!60!,
    minimum width=8em,
    draw=blue!40!black!90, very thick,
    text width=10em,
    minimum height=3.5em,
    text centered,
    on chain},
  every join/.style={<-, thick,shorten >=1pt},
  decoration={brace},
  tuborg/.style={decorate},
  tubnode/.style={midway, right=2pt},
}


\newcommand\T{\mathcal{T}}

\begin{frame}{Pfadzerlegung}
\begin{itemize}
\item Anwendbar auf beliebigen Baum $\T$
\item Erzeuge neuen Baum $\T^c$. Knoten in $\T^c$ sind Pfade in $\T$
\item Konstruktion:
\begin{enumerate}
\item W"ahle Pfad $p$ von Wurzel zu einem Blatt
\item Entferne $p$
\item Berechne rekursiv Dekompositionen $\T^c_1, \T^c_2, \ldots, \T^c_n$ der "ubrigen Teilb"aume,
geordnet nach Level der Teilbaumwurzel in $T$
\item $\T^c$ hat Wurzel $p$ mit Kindern $\T^c_1, \ldots, \T^c_n$.
\end{enumerate}
\end{itemize}
\end{frame}

\begin{frame}{Pfadzerlegung (Beispiel)}

\begin{columns}[T] % align columns
\begin{column}{.48\textwidth}

\centering
$\T$\\[1em]
%\begin{tikzpicture}[
  path11/.style={edge from parent/.style={black,thin,draw},circle,draw=black,thin,text=black},
  path00/.style={edge from parent/.style={black!20,thin,draw},circle,draw=black!20,thin,text=black!20},
  path00thick/.style={edge from parent/.style={black!20,very thick,draw},circle,draw=black!20,thin,text=black!20},
  path01/.style={edge from parent/.style={black!20,thin,draw},circle,draw=black,thin,text=black},
  path02/.style={edge from parent/.style={black!20,thin,draw},circle,draw=red,very thick,text=black},
  path22/.style={edge from parent/.style={red,very thick,draw},circle,draw=red,very thick,text=black},
  p1root/.style={path11}, p1/.style={path11},
  p2root/.style={path11}, p2/.style={path11},
  p3root/.style={path11}, p3/.style={path11},
  p4root/.style={path11}, p4/.style={path11},
  p5root/.style={path11}, p5/.style={path11},
  p6root/.style={path11}, p6/.style={path11},
]
\only<2->{\tikzset{
  p1root/.style={path22},
  p1/.style={path22},
}}
\only<3->{\tikzset{
  p1root/.style={path00},
  p1/.style={path00thick},
  p2root/.style={path01},
  p3root/.style={path01},
  p5root/.style={path01},
  p6root/.style={path01},
}}
\only<4->{\tikzset{
  p2root/.style={path02},
  p2/.style={path22},
}}
\only<5->{\tikzset{
  p2root/.style={path00},
  p2/.style={path00thick},
}}
\only<6->{\tikzset{
  p3root/.style={path02},
  p3/.style={path22},
}}
\only<7->{\tikzset{
  p3root/.style={path00},
  p3/.style={path00thick},
  p4root/.style={path01},
}}
\only<8->{\tikzset{
  p4root/.style={path02},
  p4/.style={path22},
}}
\only<9->{\tikzset{
  p4root/.style={path00},
  p4/.style={path00thick},
}}
\only<10->{\tikzset{
  p5root/.style={path02},
  p5/.style={path22},
}}
\only<11->{\tikzset{
  p5root/.style={path00},
  p5/.style={path00thick},
}}
\only<12->{\tikzset{
  p6root/.style={path02},
  p6/.style={path22},
}}
\only<13->{\tikzset{
  p6root/.style={path00},
  p6/.style={path00thick},
}}
\node[circle,draw,p1root](z){$a_1$}
  child[p1]{node[circle,draw]{$b_1$}
    child[p5root]{node[circle,draw]{$c_1$}}
    child[p1]{node[circle,draw]{$c_2$}
      child[p1]{node[circle,draw]{$d_1$}}
      child[p6root]{node[circle,draw]{$d_2$}}
    }
  }
  child[p2root]{node[circle,draw]{$b_2$}}
  child[p3root]{node[circle,draw]{$b_3$}
    child[p3]{node[circle,draw]{$c_3$}}
    child[p4root]{node[circle,draw]{$c_4$}}
  };
\end{tikzpicture}


\end{column}%
\hfill%
\begin{column}{.48\textwidth}

\centering
$\T^c$\\[1em]
%\begin{tikzpicture}[
  every node/.style={draw,rectangle},
  every child/.style={edge from parent/.style={draw=none}},
  inactive/.style={black,text=black,thin,edge from parent/.style={draw=none}},
  active/.style={inactive,draw=red,very thick},
  invisible/.style={inactive,text=white,draw=white},
  p1/.style={invisible},
  p2/.style={invisible},
  p3/.style={invisible},
  p4/.style={invisible},
  p5/.style={invisible},
  p6/.style={invisible},
  split/.style={rectangle split,rectangle split horizontal},
  conn/.style={out=270,in=90}
]
\only<2->{\tikzset{
  p1/.style={active},
}}
\only<3->{\tikzset{
  p1/.style={inactive},
}}
\only<4->{\tikzset{
  p2/.style={active},
}}
\only<5->{\tikzset{
  p2/.style={inactive},
}}
\only<6->{\tikzset{
  p3/.style={active},
}}
\only<7->{\tikzset{
  p3/.style={inactive},
}}
\only<8->{\tikzset{
  p4/.style={active},
}}
\only<9->{\tikzset{
  p4/.style={inactive},
}}
\only<10->{\tikzset{
  p5/.style={active},
}}
\only<11->{\tikzset{
  p5/.style={inactive},
}}
\only<12->{\tikzset{
  p6/.style={active},
}}
\only<13->{\tikzset{
  p6/.style={inactive},
}}
\node[draw,p1,split,rectangle split parts=4](p1){
  $a_1$
  \nodepart{two} $b_1$
  \nodepart{three} $c_2$
  \nodepart{four} $d_1$
}
  child[p2]{node(p2){$b_2$}}
  child[p3]{
    node[split,rectangle split parts=2](p3){
    $b_3$
    \nodepart{two} $c_3$
    }
    child[p4]{node(p4){$c_4$}}
  }
  child[p5]{node(p5){$c_1$}}
  child[p6]{node(p6){$d_2$}};

  \draw[conn,p2] (p1.one split south) to (p2.north);
  \draw[conn,p3] (p1.one split south) to (p3.north);
  \draw[conn,p5] (p1.two split south) to (p5.north);
  \draw[conn,p6] (p1.three split south) to (p6.north);
  \draw[conn,p4] (p3.one split south) to (p4.north);
\end{tikzpicture}


\end{column}%
\end{columns}

\end{frame}

\begin{frame}{Pfadzerlegung}
\begin{itemize}
\item Wie entscheiden, welcher Pfad der n"achste Knoten wird?
\item Zwei grundlegende Taktiken
\begin{itemize}
\item Pfad zu linkestem Blatt im Baum
\item Folge \emph{schweren Kanten}: Immer in den gr"o"sten Teilbaum absteigen
$\rightarrow \T^c$ hat Tiefe $\mathcal{O}(\log |\T|)$
\end{itemize}
\end{itemize}
\end{frame}

\newcommand{\edgelabel}[2]{edge from parent node[draw=none,#1]{\lstinline|#2|}}

\begin{frame}{Pfadzerlegung eines Tries}
\begin{itemize}
\item Betrachte Patricia-Trie mit Knotenlabels
\end{itemize}

\begin{columns}[T] % align columns
\begin{column}{.48\textwidth}

\centering
\begin{tikzpicture}[
  every node/.style={rectangle,draw,black,text=black,thin,
    edge from parent/.style={draw,black,thin}},
]
\node(z){\lstinline|r|}
  child{
    node{\lstinline|om|}
    child{
      node{\lstinline|ulus|}
      \edgelabel{left}{u}
    }
    child{node{\lstinline|an|}
      child{
        node {\lstinline|e|}
        \edgelabel{left}{e}
      }
      child{
        node{\lstinline|us|}
        \edgelabel{right}{u}
      }
      \edgelabel{right}{a}
    }
    \edgelabel{left}{o}
  }
  child{
    node{\lstinline|emus|}
    \edgelabel{left}{e}
  }
  child{node{\lstinline|ub|}
    child{
      node{\lstinline|en|}
      \edgelabel{left}{e}
    }
    child{
      node{\lstinline|icon|}
      \edgelabel{right}{i}
    }
    \edgelabel{right}{u}
  };
\end{tikzpicture}

\end{column}%
\hfill%
\begin{column}{.48\textwidth}

\centering
\begin{tikzpicture}[
  every node/.style={rectangle,draw,black,text=black,thin,
    edge from parent/.style={draw,black,thin}},
]
\node[rectangle split,rectangle split parts=4, rectangle split horizontal](z){
  \lstinline|r|
  \nodepart{two} \lstinline|om|
  \nodepart{three} \lstinline|an|
  \nodepart{four} \lstinline|us|
  }
  child{
    node{\lstinline|emus|}
  }
  child{
    node{\lstinline|uben|}
    child{
      node{\lstinline|icon|}
    }
  }
  child{
    node{\lstinline|ulus|}
  }
  child{
    node{\lstinline|us|}
  }
  ;
\end{tikzpicture}

\end{column}
\end{columns}

\end{frame}

\end{document}
